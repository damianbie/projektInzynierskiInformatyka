\section{Wprowadzenie}

Obserwując rynek możemy zauważyć ciągły wzrost liczby urządzeń będących coraz bardziej inteligentnych
i odpornych na ciągle zmieniające się otoczenie.

Celem projektu inżynierskiego jest zbudowanie robota śledzącego wskazany obiekt, który jest wykryty przez zamontowaną na nim kamerę.
Robot będzie sterowany przez komputer przetwarzający obraz przy pomocy odpowiednio wytrenowanej
głębokiej sieci neuronowej. Zadaniem sieci będzie rozpoznanie i zlokalizowanie wyznaczonego obiektu, 
a następnie przekazanie jego pozycji algorytmu sterującego. Dalsza część programu
wyśle odpowiednio spreparowane komendy do robota, aby ten ustawił karetkę z kamerą na obiektem.

Głównym powodem realizacji takiego tematu jest chęć poznania działania i trenowania głębokich sieci 
neuronowych rozpoznających obiekty na obrazach.

\textbf{Omówienie rozdziałów}

Mając na uwadze zakres pracy i cel projektu, jej treść podzielono na szereg rozdziałów. 
Rozdział pierwszy ogólnie omawia typy i zasadę działania sieci neuronowych. 
Rozdział drugi skupia się na przeglądzie literatury, a w szczególności na opisie istniejących rozwiązań
wykorzystujących rozpoznawanie obiektów przy pomocy głębokiej sieci neuronowej.
Rozdział trzeci szczegółowo omawia proces budowy wykorzystywanego dalej robota.
Rozdział czwarty opisuje proces pozyskania danych i trenowania sieci rozpoznającej obiekty.
Rozdział piąty omawia utworzony program wykorzystujący wytrenowaną sieć neuronową, cały proces sterowania i 
sposób połączenia wszystkich systemów w działający projekt.
Ostatni rozdział przedstawia testy finalnej wersji programu oraz ich krótki opis. 
