\section{Wprowadzenie}

Obserwując rynek i możemy zauważyć ciągły wzrost urządzeń, będących coraz bardziej inteligentnych
i odpornych na ciągle zmieniające się otoczenie. 

Celem projektu inżynierskiego jest zbudowanie robota śledzącego poprzez kamerę wskazany obiekt.
Robot będzie sterowany przez komputer przetwarzający obraz przy pomocy odpowiednio wytrenowanej
głębokiej sieci neuronowej. Zadaniem sieci będzie rozpoznanie i zlokalizowanie wyznaczonego obiektu
a następnie przekazanie jego pozycji na obrazie dalej do algorytmu. Dalsza część programu
wyślę odpowiednio spreparowane komendy do robota aby ten ustawił się na obiektem.

Głównym powodem realizacji takiego tematu jest chęć poznania działania i trenowania głębokich sieci 
neuronowych rozpoznających obiekty na obrazach.

\textbf{Omówienie rozdziałów}


Rozdział pierwszy - omawia działanie głębokich sieci neuronowych


Rozdział drugi - przegląd literatury 


Rozdział trzeci - zawiera opis procesu trenowania sieci rozpoznającej wskazany obiekt


Rozdział czwarty - opisuje realizacje pozostałej części programu 

Rozdział piąty - przedstawia modyfikacje i komunikacje z wykorzystanym robotem 
odpowiadającym za przemieszczenie kamery. 

Rozdział szósty - przedstawia przeprowadzone opis i wyniki przeprowadzonych testów,
różne warunki pracy algorytmu.